\documentclass{article}

\usepackage{amsmath}
\usepackage{alvinnotes}
\usepackage{lipsum} % filler

\title{Test}
\author{Alvin Kim}

\begin{document}
    \maketitle

    \tableofcontents

    \begin{problem}[Physical Force of the Thing]
        Given the equation
        \[F = ma\]
        how much wood could a woodchuck chuck if a wood chuck could chuck wood?
        \begin{solution}
            Start with this.
            \begin{align}
                F &= ma & \text{divide by \(a\)} \\
                \frac{F}{a} &= \boxed{m} & \text{Wow} 
            \end{align}
            \boxed{This} is the final answer.
        \end{solution}
    \end{problem}

    \begin{problem}
        Where?
        \begin{solution}
            The question is not ``Where?'' but ``When?''.
        \end{solution}
    \end{problem}

    \begin{problem}[Multi-Part-Problem]
        Consider the equation
        \[x = 1\]

        \begin{subproblem}
            Find the derivative
            \begin{solution}
                Ok
            \end{solution}
        \end{subproblem}

        \begin{subproblem}
            Find the idk
        \end{subproblem}

        \begin{subproblem}
            Find the thing
        \end{subproblem}
    \end{problem}

    \begin{problem}
        \lipsum[1]

        \begin{solution}
            \lipsum[2]
        \end{solution}
    \end{problem}

    \begin{problem}
        \lipsum[3]

        \begin{subproblem}
            \lipsum[4]
            \begin{solution}
                \lipsum[5]
            \end{solution}
        \end{subproblem}
        \begin{subproblem}
            \lipsum[6]
            \begin{solution}
                \lipsum[7]
            \end{solution}
        \end{subproblem}
        \begin{subproblem}
            \lipsum[8]
            \begin{solution}
                \lipsum[9]
            \end{solution}
        \end{subproblem}
    \end{problem}
\end{document}